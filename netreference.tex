\documentclass[12pt,letterpaper]{article}
\usepackage[T1]{fontenc}
\usepackage[margin=1in]{geometry}
\usepackage{multicol}
\usepackage{lmodern}
\usepackage{textcomp}
\usepackage[compact]{titlesec}

\newcommand{\texttit}[1]{\texttt{\textit{#1}}}
\newenvironment{itemize*}%
	{\begin{itemize}%
		\setlength{\itemsep}{0pt}%
		\setlength{\parskip}{0pt}%
		\footnotesize%
	}%
	{\end{itemize}}


\setlength{\parindent}{0pt}
\setlength{\parskip}{-10pt}
\thispagestyle{empty}

\begin{document}

\noindent
\framebox{
	\parbox{\textwidth}{
		\begin{centering}
		\begin{scriptsize}
		CS306 \hfill
		{\small Web Design} \hfill
		Notes \\
		2016--2017 Term 3 \\
		\hfill \\
		\textbf{\large Reference Sheet: Node.js and Network Communication} \\
		Name: \hrulefill
		\quad
		Section: \hrulefill
		\end{scriptsize}
		\end{centering}
	}
} \\

\section*{Debugging}

\texttt{console.log(message)}
\begin{itemize*}
\item[] \texttt{message :\ string}
\item[] Writes a line of text to the browser console or terminal in which Node.js is running
\end{itemize*}

\section*{Node.js: Server-side JavaScript}

\texttt{require(module)}
\begin{itemize*}
\item[] \texttt{module :\ string} \qquad Returns \texttt{object}
\item[] Loads and returns the library module with the specified name
\end{itemize*}

\subsection*{Module \texttt{http}}

\vspace{-18pt}
\begin{multicols}{2}
\texttt{createServer(handler)}
\begin{itemize*}
\item[] \texttt{handler :\ \textnormal{function}} \qquad Returns \texttt{object}
\item[] Creates a \texttit{server} whose incoming requests are processed by the supplied function:
\end{itemize*}
\columnbreak
\texttt{handler(\textit{request}, \textit{response})}
\begin{itemize*}
\item[] \texttt{\textit{request} :\ object \quad \textit{response} :\ object}
\item[] Called each time a new connection is received; it will be held open until explicitly closed
\end{itemize*}
\end{multicols}

\texttt{\textit{server}.listen(port)}
\begin{itemize*}
\item[] \texttt{port :\ number}
\item[] Starts accepting connections on the specified port.
	Use port 80 to handle \texttt{http://localhost}, or \textit{e.g.}\ port 1024 for \texttt{http://localhost:1024} (must use this or a higher port on Mac OS).
\end{itemize*}

\texttt{\textit{request}.url}
\begin{itemize*}
\item[] \texttt{string}
\item[] The path to the requested resource (\textit{e.g.}\ file).
	To strip the leading slash: \texttt{\textit{request}.url.substring(1)}
\end{itemize*}

\texttt{\textit{response}.end(message)}
\begin{itemize*}
\item[] \texttt{message :\ string}
\item[] Transmits the specified reply and closes the connection
\end{itemize*}

\subsection*{Module \texttt{url}}

\texttt{parse(url, arguments)}
\begin{itemize*}
\item[] \texttt{url :\ string \quad arguments :\ boolean} \qquad Returns \texttt{object}
\item[] Returns an object that breaks the URL into pieces given by string properties (\textit{e.g.}\ \texttt{pathname} for the path to the resource).
	If \texttt{arguments} was \texttt{true}, it also contains an object \texttt{query} with a string property for each named argument (\textit{e.g.}\ \texttt{name1} and \texttt{name2} in \texttt{/filename.html?name1=value1\&name2=value2}).
\end{itemize*}

\subsection*{Module \texttt{fs}}

\texttt{existsSync(path)}
\begin{itemize*}
\item[] \texttt{path :\ string} \qquad Returns \texttt{boolean}
\item[] Reports whether there exists a file at the specified path
\end{itemize*}

\texttt{readFileSync(path)}
\begin{itemize*}
\item[] \texttt{path :\ string} \qquad Returns \texttt{string}
\item[] Returns a string containing the contents of the file at the specified path.
	Crashes if it does not exist!
\end{itemize*}

\section*{XMLHttpRequest: Client-side networking in JavaScript}

\texttt{new XMLHttpRequest()}
\begin{itemize*}
\item[] Returns \texttt{object}
\item[] Constructs a bidirectional network \texttit{connection}
\end{itemize*}

\texttt{\textit{connection}.open(method, url)}
\begin{itemize*}
\item[] \texttt{method :\ string \quad url :\ string}
\item[] To behave as if the \texttt{url} was entered in the browser's location bar, set \texttt{method} to \texttt{\textquotesingle GET\textquotesingle}.
\end{itemize*}

\texttt{\textit{connection}.send()}
\begin{itemize*}
\item[] Transmits the request to the server.
	Call after \texttt{open()}.
\end{itemize*}

\texttt{\textit{connection}.onreadystatechange}
\begin{itemize*}
\item[] Set to a handler function to be called each time progress is made.
	Once \texttt{\textit{connection}.readyState} is \texttt{XMLHttpRequest.DONE}, the string \texttt{\textit{connection}.responseText} contains the server's reply.
\end{itemize*}

\end{document}
