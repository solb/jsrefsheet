\documentclass[12pt,letter]{article}
\usepackage{amsfonts}
\usepackage[T1]{fontenc}
\usepackage[margin=1in]{geometry}
\usepackage{lmodern}
\usepackage{multicol}
\usepackage{textcomp}
\usepackage[compact]{titlesec}

\newcommand{\texttb}[1]{\texttt{\textbf{#1}}}
\setlength{\parindent}{0pt}
\thispagestyle{empty}

\begin{document}

\noindent
\framebox{
	\parbox{\textwidth}{
		\begin{centering}
		\begin{scriptsize}
		CS306 \hfill
		{\small Web Design} \hfill
		Notes \\
		2016-2017 Term 2 \\
		\hfill \\
		\textbf{\large Reference Sheet: JavaScript and DOM} \\
		Name: \hrulefill
		\quad
		Section: \hrulefill
		\end{scriptsize}
		\end{centering}
	}
}

\hfill

\begin{multicols}{2}
\framebox[\linewidth]{
	\parbox{\linewidth}{
		\section*{Types}
		Declare a weakly-typed variable: \texttb{var} \textit{variable}

		\begin{tabular}{c c}
		\underline{Type} & \underline{Literal(s)} \\
		\texttt{undefined} & \texttb{undefined} * \\
		\texttt{boolean} & \texttb{true} or \texttb{false} \\
		\texttt{number} & \texttt{0}, \texttt{-1}, \texttt{2.718}, \texttb{NaN}, etc. \\
		\texttt{string} & \texttt{\textquotesingle}\textit{text}\texttt{\textquotesingle} or \texttt{"}\textit{text}\texttt{"} $^\dagger$ \\
		\multicolumn{2}{l}{\hspace{10pt}\texttt{object} \texttt{\{}\textit{variable}\texttt{:} \textit{value}\texttt{,} \textit{var}\texttt{:} \textit{val}\texttt{,} \textellipsis\texttt{\}}} \\
		& or \texttb{null} \\
		\texttt{Array} & \texttt{[}\textit{value}\texttt{,} \textit{value}\texttt{,} \textellipsis\texttt{]} $^\ddagger$ \\
		\end{tabular}

		{\small
			\ * \hspace{-6pt} All variables are initially set to \texttb{undefined} until a different value is assigned to them.

			$^{\ \dagger}$ Certain characters within \textit{text} must be escaped, \textit{e.g.} \texttt{\textbackslash n}, \texttt{\textbackslash\textquotesingle}, \texttt{\textbackslash"}, and \texttt{\textbackslash\textbackslash}

			$^{\ \ddagger}$ Subtype of object that can also be created by calling constructor: \texttt{\textbf{new} Array(}\textit{length}\texttt{)}
		}
	}
}

\hfill

\framebox[\linewidth]{
	\parbox{\linewidth}{
		\section*{Casting}
		Each cast function expects a single argument and returns a value of the type being cast to.

		\vspace{-12pt}
		\begin{center}
		\begin{tabular}{c c}
		\underline{Type} & \underline{Cast function} \\
		\texttt{boolean} & \texttt{Boolean()} \\
		\texttt{number} & \texttt{Number()} $^\dagger$ \\
		& \texttt{parseFloat()} \\
		& \texttt{parseInt()} $^\ddagger$ \\
		\texttt{string} & \texttt{String()} \\
		\end{tabular}
		\end{center}

		\vspace{-12pt}
		{\small
			$^{\ \dagger}$ Returns \texttt{NaN} as soon as it encounters a non-digit character in a \texttt{string}. Use \texttt{parseFloat()} instead to avoid this behavior.

			$^{\ \ddagger}$ In addition to casting, is often used on \texttt{number}s to truncate their fractional components.
		}
	}
}

\hfill

\framebox[\linewidth]{
	\parbox{\linewidth}{
		\section*{Conditionals}
		\vspace{-6pt}
		\texttt{\textbf{if}(}\textit{predicate}\texttt{) \{\}} \\
		\texttt{\textbf{else if}(}\textit{predicate}\texttt{) \{\}} \\
		\texttt{\textbf{else} \{\}}

		{\small
			\textit{predicate}s are of type \texttt{boolean}. There may be any number of \texttb{else if}s, but at most one \texttb{else}.
		}
	}
}

\columnbreak

\framebox[\linewidth]{
	\parbox{\linewidth}{
		\section*{Operators \& coercion}
		In order of execution (adjust w/ \texttt{()}s):

		\begin{tabular}{c c c c c}
		\underline{Op} & \underline{Left} & \underline{Right} & \hspace{-6pt}\underline{Coerce}\hspace{-6pt} & \underline{Result} \\
		\texttt{.} & \texttt{object} & \textit{variable} & \checkmark & any \\
		\texttt{[]} & \texttt{object} & \texttt{string} & \checkmark & any \\
		\hline
		\texttt{++} & \textit{variable} & N/A & \checkmark & \texttt{number} \\
		\texttt{-{}-} & \textit{variable} & N/A & \checkmark & \texttt{number} \\
		\hline
		\texttt{!} & N/A & \texttt{boolean} & \checkmark & \texttt{boolean} \\
		\hline
		\texttt{\textasteriskcentered\textasteriskcentered} & \texttt{number} & \texttt{number} & \checkmark & \texttt{number} \\
		\hline
		\texttt{\textasteriskcentered} & \texttt{number} & \texttt{number} & \checkmark & \texttt{number} \\
		\texttt{/} & \texttt{number} & \texttt{number} & \checkmark & \texttt{number} \\
		\texttt{\%} & \texttt{number} & \texttt{number} & \checkmark & \texttt{number} \\
		\hline
		\texttt{+} & ? & ? & $\dagger$ & input \\
		\texttt{-} & \texttt{number} & \texttt{number} & \checkmark & \texttt{number} \\
		\hline
		\texttt{<} & any & any & $\ddagger$ & \texttt{boolean} \\
		\hline
		\texttt{==} & any & any & $\wedge$ & \texttt{boolean} \\
		\texttt{===} & any & any & & \texttt{boolean} \\
		\texttt{!==} & any & any & & \texttt{boolean} \\
		\hline
		\texttt{\&\&} & \texttt{boolean} & \texttt{boolean} & \checkmark & \texttt{boolean} \\
		\hline
		\texttt{||} & \texttt{boolean} & \texttt{boolean} & \checkmark & \texttt{boolean} \\
		\hline
		\texttt{=} & \textit{variable} & any & & input \\
		\end{tabular}

		{\small
			Note that \texttt{+=}, \texttt{-=}, \texttt{*=}, \texttt{/=}, \texttt{\%=}, and \texttt{**=} are also valid assignment operators.

			$^{\ \dagger}$ Supports both \texttt{number}s and \texttt{string}s, but coercion prefers to convert to \texttt{string}.

			$^{\ \ddagger}$ And \texttt{>}, \texttt{<=}, \texttt{>=}. Coercion prefers \texttt{number}.

			$^{\ \wedge}$ And \texttt{!=}. Coercion prefers \texttt{number}. \texttt{object}s are tested for being the \textbf{same} instance.
		}
	}
}

\hfill

\framebox[\linewidth]{
	\parbox{\linewidth}{
		\section*{Loops}
		\vspace{-6pt}
		\texttt{\textbf{while}(}\textit{predicate}\texttt{) \{\}} \\
		\texttt{\textbf{for}(}\textit{initializer}\texttt{; }\textit{predicate}\texttt{; }\textit{increment}\texttt{) \{\}} \\
		{\scriptsize
			\hspace*{1em} \textit{e.g.} \texttt{\textbf{for}(\textbf{var} count = 1; count <= 5; count++) \{\}}

			\hspace*{1em} \textit{or} \texttt{\textbf{var} count = 0} to start from 0; \texttt{count+=2} to count by 2
		}

		{\small
			\textit{predicate}s are of type \textbf{boolean}. Both types of loop run the body (everything between the curly braces) as long as \textit{predicate} is \texttt{true}, stopping as soon as it becomes \texttt{false}. Any and all parts of the \texttb{for} loop header may be omitted.
		}
	}
}
\end{multicols}

\end{document}
